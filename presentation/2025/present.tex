\let\RaggedRight\raggedright
\let\raggedright\relax
\documentclass[utf8,9pt,mathserif,usepdftitle=false]{beamer}

\usepackage{graphicx}%
\usepackage{float}%
\usepackage{wrapfig}%

\let\RaggedRight\raggedright
\let\raggedright\relax
\documentclass[utf8,9pt,mathserif,usepdftitle=false]{beamer}

\usepackage{graphicx}%
\usepackage{float}%
\usepackage{wrapfig}%

\let\RaggedRight\raggedright
\let\raggedright\relax
\documentclass[utf8,9pt,mathserif,usepdftitle=false]{beamer}

\usepackage{graphicx}%
\usepackage{float}%
\usepackage{wrapfig}%

\let\RaggedRight\raggedright
\let\raggedright\relax
\documentclass[utf8,9pt,mathserif,usepdftitle=false]{beamer}

\usepackage{graphicx}%
\usepackage{float}%
\usepackage{wrapfig}%

\include{present.daf}

% \setbeameroption{show notes on second screen}
\setbeameroption{hide notes}
\mode<presentation>
{
  \usetheme{default}
  %\usecolortheme{default} % dove, beaver
  \usecolortheme{whale}
  %\usefonttheme{serif}
}

\hypersetup{%
  pdfinfo={%
    Title={Защита курсовых работ, ИГУ},%
    Subject={О воспроизводимости результатов численных решений уравнения осцилляции нейтрино в среде},%
    Author={Данеко Илья},%
    Keywords={neutrino oscillation in matter;quality properties}%
  }
}

\title{О воспроизводимости результатов численных решений уравнения осцилляции нейтрино в среде}%
\author{Данеко И.И.}
\date[ИГУ, 2025]{%
  20 октября 2025\\%
  \vspace*{10ex}%
  \begin{flushright}
    \small
    Научный руководитель: Ломов В.П.\par
  \end{flushright}
  {\vspace*{7ex}
    \footnotesize%
    Иркутск, ФГБОУ ВО ИГУ\par%
  } }

\begin{document}

\begin{frame}
  \titlepage
\end{frame}

\begin{frame}
  \frametitle{Введение}%
  %%%
  % TODO:
  \begin{itemize}
  \item<1-> Нейтрино...
  \item<2-> Осцилляции в среде...
  \item<3-> Проблемы численных расчётов.
  \end{itemize}
  \onslide<4->%
  В данной работе мы ...
\end{frame}

\begin{frame}
  \frametitle{Осцилляции нейтрино}%
  %%%
  Что такое нейтрино

  \onslide<2->%
  Массивные нейтрино...

  \onslide<3->%
  Переход от одного вида к другому
\end{frame}

\begin{frame}
  \frametitle{Осцилляции нейтрино в веществе}%
  %%%
  Уравнение осцилляций

  \onslide<2->%
  Профиль плотности для солнечной модели

  \onslide<3->%
  ...
\end{frame}

\begin{frame}
  \frametitle{Наблюдаемые}%
  %%%
  Вероятность выживания

  \onslide<2->
  Теоретическая формула
\end{frame}

\begin{frame}
  \frametitle{Качественные свойства решения}%
  %%%
  %\includegraphics{graph1}
  График 1
\end{frame}

\begin{frame}
  \frametitle{Качественные свойства решения}%
  %%%
  %\includegraphics{graph2}
  График 2
\end{frame}

\begin{frame}
  \frametitle{Качественные свойства решения}%
  %%%
  %\includegraphics{graph3}
  График 3
\end{frame}

\begin{frame}
  \frametitle{Качественные свойства решения}%
  %%%
  %\includegraphics{quality}
  Контроль качества решения:
  График 4
\end{frame}

\begin{frame}
  \frametitle{Заключение}%
  %%%
  В данной работы мы получили
  \begin{itemize}
  \item<1-> что-то хорошее
  \item<2-> не очень хорошее, но можно сделать в будущем (лучше?)
  \end{itemize}
\end{frame}

\begin{frame}
  \frametitle{КОНЕЦ}%
  \LARGE%
  \centering%
  \bfseries%
  СПАСИБО ЗА ВНИМАНИЕ%
\end{frame}

\begin{frame}
  \frametitle{Дополнительно}%
  %%%
  \centering%
  ДОПОЛНИТЕЛЬНЫЙ МАТЕРИАЛ
\end{frame}

\end{document}

%%% Local Variables:
%%% mode: latex
%%% fill-column: 80
%%% TeX-master: t
%%% TeX-PDF-mode: t
%%% End:
%%% vim: syn=tex ft=tex tw=80 ts=2 sw=2 et:


% \setbeameroption{show notes on second screen}
\setbeameroption{hide notes}
\mode<presentation>
{
  \usetheme{default}
  %\usecolortheme{default} % dove, beaver
  \usecolortheme{whale}
  %\usefonttheme{serif}
}

\hypersetup{%
  pdfinfo={%
    Title={Защита курсовых работ, ИГУ},%
    Subject={О воспроизводимости результатов численных решений уравнения осцилляции нейтрино в среде},%
    Author={Данеко Илья},%
    Keywords={neutrino oscillation in matter;quality properties}%
  }
}

\title{О воспроизводимости результатов численных решений уравнения осцилляции нейтрино в среде}%
\author{Данеко И.И.}
\date[ИГУ, 2025]{%
  20 октября 2025\\%
  \vspace*{10ex}%
  \begin{flushright}
    \small
    Научный руководитель: Ломов В.П.\par
  \end{flushright}
  {\vspace*{7ex}
    \footnotesize%
    Иркутск, ФГБОУ ВО ИГУ\par%
  } }

\begin{document}

\begin{frame}
  \titlepage
\end{frame}

\begin{frame}
  \frametitle{Введение}%
  %%%
  % TODO:
  \begin{itemize}
  \item<1-> Нейтрино...
  \item<2-> Осцилляции в среде...
  \item<3-> Проблемы численных расчётов.
  \end{itemize}
  \onslide<4->%
  В данной работе мы ...
\end{frame}

\begin{frame}
  \frametitle{Осцилляции нейтрино}%
  %%%
  Что такое нейтрино

  \onslide<2->%
  Массивные нейтрино...

  \onslide<3->%
  Переход от одного вида к другому
\end{frame}

\begin{frame}
  \frametitle{Осцилляции нейтрино в веществе}%
  %%%
  Уравнение осцилляций

  \onslide<2->%
  Профиль плотности для солнечной модели

  \onslide<3->%
  ...
\end{frame}

\begin{frame}
  \frametitle{Наблюдаемые}%
  %%%
  Вероятность выживания

  \onslide<2->
  Теоретическая формула
\end{frame}

\begin{frame}
  \frametitle{Качественные свойства решения}%
  %%%
  %\includegraphics{graph1}
  График 1
\end{frame}

\begin{frame}
  \frametitle{Качественные свойства решения}%
  %%%
  %\includegraphics{graph2}
  График 2
\end{frame}

\begin{frame}
  \frametitle{Качественные свойства решения}%
  %%%
  %\includegraphics{graph3}
  График 3
\end{frame}

\begin{frame}
  \frametitle{Качественные свойства решения}%
  %%%
  %\includegraphics{quality}
  Контроль качества решения:
  График 4
\end{frame}

\begin{frame}
  \frametitle{Заключение}%
  %%%
  В данной работы мы получили
  \begin{itemize}
  \item<1-> что-то хорошее
  \item<2-> не очень хорошее, но можно сделать в будущем (лучше?)
  \end{itemize}
\end{frame}

\begin{frame}
  \frametitle{КОНЕЦ}%
  \LARGE%
  \centering%
  \bfseries%
  СПАСИБО ЗА ВНИМАНИЕ%
\end{frame}

\begin{frame}
  \frametitle{Дополнительно}%
  %%%
  \centering%
  ДОПОЛНИТЕЛЬНЫЙ МАТЕРИАЛ
\end{frame}

\end{document}

%%% Local Variables:
%%% mode: latex
%%% fill-column: 80
%%% TeX-master: t
%%% TeX-PDF-mode: t
%%% End:
%%% vim: syn=tex ft=tex tw=80 ts=2 sw=2 et:


% \setbeameroption{show notes on second screen}
\setbeameroption{hide notes}
\mode<presentation>
{
  \usetheme{default}
  %\usecolortheme{default} % dove, beaver
  \usecolortheme{whale}
  %\usefonttheme{serif}
}

\hypersetup{%
  pdfinfo={%
    Title={Защита курсовых работ, ИГУ},%
    Subject={О воспроизводимости результатов численных решений уравнения осцилляции нейтрино в среде},%
    Author={Данеко Илья},%
    Keywords={neutrino oscillation in matter;quality properties}%
  }
}

\title{О воспроизводимости результатов численных решений уравнения осцилляции нейтрино в среде}%
\author{Данеко И.И.}
\date[ИГУ, 2025]{%
  20 октября 2025\\%
  \vspace*{10ex}%
  \begin{flushright}
    \small
    Научный руководитель: Ломов В.П.\par
  \end{flushright}
  {\vspace*{7ex}
    \footnotesize%
    Иркутск, ФГБОУ ВО ИГУ\par%
  } }

\begin{document}

\begin{frame}
  \titlepage
\end{frame}

\begin{frame}
  \frametitle{Введение}%
  %%%
  % TODO:
  \begin{itemize}
  \item<1-> Нейтрино...
  \item<2-> Осцилляции в среде...
  \item<3-> Проблемы численных расчётов.
  \end{itemize}
  \onslide<4->%
  В данной работе мы ...
\end{frame}

\begin{frame}
  \frametitle{Осцилляции нейтрино}%
  %%%
  Что такое нейтрино

  \onslide<2->%
  Массивные нейтрино...

  \onslide<3->%
  Переход от одного вида к другому
\end{frame}

\begin{frame}
  \frametitle{Осцилляции нейтрино в веществе}%
  %%%
  Уравнение осцилляций

  \onslide<2->%
  Профиль плотности для солнечной модели

  \onslide<3->%
  ...
\end{frame}

\begin{frame}
  \frametitle{Наблюдаемые}%
  %%%
  Вероятность выживания

  \onslide<2->
  Теоретическая формула
\end{frame}

\begin{frame}
  \frametitle{Качественные свойства решения}%
  %%%
  %\includegraphics{graph1}
  График 1
\end{frame}

\begin{frame}
  \frametitle{Качественные свойства решения}%
  %%%
  %\includegraphics{graph2}
  График 2
\end{frame}

\begin{frame}
  \frametitle{Качественные свойства решения}%
  %%%
  %\includegraphics{graph3}
  График 3
\end{frame}

\begin{frame}
  \frametitle{Качественные свойства решения}%
  %%%
  %\includegraphics{quality}
  Контроль качества решения:
  График 4
\end{frame}

\begin{frame}
  \frametitle{Заключение}%
  %%%
  В данной работы мы получили
  \begin{itemize}
  \item<1-> что-то хорошее
  \item<2-> не очень хорошее, но можно сделать в будущем (лучше?)
  \end{itemize}
\end{frame}

\begin{frame}
  \frametitle{КОНЕЦ}%
  \LARGE%
  \centering%
  \bfseries%
  СПАСИБО ЗА ВНИМАНИЕ%
\end{frame}

\begin{frame}
  \frametitle{Дополнительно}%
  %%%
  \centering%
  ДОПОЛНИТЕЛЬНЫЙ МАТЕРИАЛ
\end{frame}

\end{document}

%%% Local Variables:
%%% mode: latex
%%% fill-column: 80
%%% TeX-master: t
%%% TeX-PDF-mode: t
%%% End:
%%% vim: syn=tex ft=tex tw=80 ts=2 sw=2 et:


% \setbeameroption{show notes on second screen}
\setbeameroption{hide notes}
\mode<presentation>
{
  \usetheme{default}
  %\usecolortheme{default} % dove, beaver
  \usecolortheme{whale}
  %\usefonttheme{serif}
}

\hypersetup{%
  pdfinfo={%
    Title={Защита курсовых работ, ИГУ},%
    Subject={О воспроизводимости результатов численных решений уравнения осцилляции нейтрино в среде},%
    Author={Данеко Илья},%
    Keywords={neutrino oscillation in matter;quality properties}%
  }
}

\title{О воспроизводимости результатов численных решений уравнения осцилляции нейтрино в среде}%
\author{Данеко И.И.}
\date[ИГУ, 2025]{%
  20 октября 2025\\%
  \vspace*{10ex}%
  \begin{flushright}
    \small
    Научный руководитель: Ломов В.П.\par
  \end{flushright}
  {\vspace*{7ex}
    \footnotesize%
    Иркутск, ФГБОУ ВО ИГУ\par%
  } }

\begin{document}

\begin{frame}
  \titlepage
\end{frame}

\begin{frame}
  \frametitle{Введение}%
  %%%
  % TODO:
  \begin{itemize}
  \item<1-> Нейтрино...
  \item<2-> Осцилляции в среде...
  \item<3-> Проблемы численных расчётов.
  \end{itemize}
  \onslide<4->%
  В данной работе мы ...
\end{frame}

\begin{frame}
  \frametitle{Осцилляции нейтрино}%
  %%%
  Что такое нейтрино

  \onslide<2->%
  Массивные нейтрино...

  \onslide<3->%
  Переход от одного вида к другому
\end{frame}

\begin{frame}
  \frametitle{Осцилляции нейтрино в веществе}%
  %%%
  Уравнение осцилляций

  \onslide<2->%
  Профиль плотности для солнечной модели

  \onslide<3->%
  ...
\end{frame}

\begin{frame}
  \frametitle{Наблюдаемые}%
  %%%
  Вероятность выживания

  \onslide<2->
  Теоретическая формула
\end{frame}

\begin{frame}
  \frametitle{Качественные свойства решения}%
  %%%
  %\includegraphics{graph1}
  График 1
\end{frame}

\begin{frame}
  \frametitle{Качественные свойства решения}%
  %%%
  %\includegraphics{graph2}
  График 2
\end{frame}

\begin{frame}
  \frametitle{Качественные свойства решения}%
  %%%
  %\includegraphics{graph3}
  График 3
\end{frame}

\begin{frame}
  \frametitle{Качественные свойства решения}%
  %%%
  %\includegraphics{quality}
  Контроль качества решения:
  График 4
\end{frame}

\begin{frame}
  \frametitle{Заключение}%
  %%%
  В данной работы мы получили
  \begin{itemize}
  \item<1-> что-то хорошее
  \item<2-> не очень хорошее, но можно сделать в будущем (лучше?)
  \end{itemize}
\end{frame}

\begin{frame}
  \frametitle{КОНЕЦ}%
  \LARGE%
  \centering%
  \bfseries%
  СПАСИБО ЗА ВНИМАНИЕ%
\end{frame}

\begin{frame}
  \frametitle{Дополнительно}%
  %%%
  \centering%
  ДОПОЛНИТЕЛЬНЫЙ МАТЕРИАЛ
\end{frame}

\end{document}

%%% Local Variables:
%%% mode: latex
%%% fill-column: 80
%%% TeX-master: t
%%% TeX-PDF-mode: t
%%% End:
%%% vim: syn=tex ft=tex tw=80 ts=2 sw=2 et:
